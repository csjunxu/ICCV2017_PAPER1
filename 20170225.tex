
\documentclass[10pt,twocolumn,letterpaper]{article}

\usepackage{iccv}
\usepackage{times}
\usepackage{epsfig}
\usepackage{graphicx}
\usepackage{amsmath}
\usepackage{amssymb}
\usepackage[numbers,sort]{natbib}
\usepackage[UTF8]{ctex}



% Include other packages here, before hyperref.

% If you comment hyperref and then uncomment it, you should delete
% egpaper.aux before re-running latex.  (Or just hit 'q' on the first latex
% run, let it finish, and you should be clear).
\usepackage[pagebackref=true,breaklinks=true,letterpaper=true,colorlinks,bookmarks=false]{hyperref}

% \iccvfinalcopy % *** Uncomment this line for the final submission

\def\iccvPaperID{****} % *** Enter the ICCV Paper ID here
\def\httilde{\mbox{\tt\raisebox{-.5ex}{\symbol{126}}}}

% Pages are numbered in submission mode, and unnumbered in camera-ready
\ificcvfinal\pagestyle{empty}\fi
\begin{document}

%%%%%%%%% TITLE
\title{\LaTeX\ Author Guidelines for ICCV Proceedings}

\author{First Author\\
Institution1\\
Institution1 address\\
{\tt\small firstauthor@i1.org}
% For a paper whose authors are all at the same institution,
% omit the following lines up until the closing ``}''.
% Additional authors and addresses can be added with ``\and'',
% just like the second author.
% To save space, use either the email address or home page, not both
\and
Second Author\\
Institution2\\
First line of institution2 address\\
{\tt\small secondauthor@i2.org}
}

\maketitle
%\thispagestyle{empty}

%%%%%%%%% ABSTRACT
\begin{abstract}
Motivated by the weighted Orthogonal Procrustes Problem, we propose a noval weighted Frobenious norm based weighted sparse coding model for non-Gaussian error modeling. We solve this model in an alternative manner. Updating of each variable has closed-form solutions and the overall model converges to a stationary point. The proposed model is applied in real image denoising problem and extensive experiments demonstrate that the proposed model can much better performance (over 1.0dB improvement on PSNR) than state-of-the-art image denoising methods, including some excellant commercial software. The noval weighted Frobeniius norm can perfectly fit the non-Gaussian property of real noise.
\end{abstract}

\section{Introduction}

Image denoising is an important problem in computer vision and image processing. It aims to remove the noise from the noisy images to obtain the laten clean image which could be used in other computer vision tasks such as image restoration and segementation. In real world, the natural images, such as color image and heperspectral image, often contain multi-channels instead of one (grayscale image). And the multi-channel images often exhibit different noise levels in different channels. For example, real noisy color images often contains different noise levels among the R, G, B channels. This is caused by the color demosaicking during the transformation from raw data to RGB images in the standard in-camera imaging pipeline. Usually, the G channel contains the least noise levels among the three channels. Hence, in order to deal with each channel more effectively, different weights should be added to different channels.
















The non-local self similarity property of images has been extensively employed in image denoising algorithms \cite{}. Among these methods, the weighted nuclear norm minimization (WNNM) achieved the state-of-the-art performance in grayscale image denoising with additive white Gaussian noise (AWGN). Though among the most effective methods, how WNNM can be extended to deal with color image denoising and real-world image denoising is still an open problem. In this paper, we proposed a multi-channel weighted nuclear norm minimization model to deal with the color image denoising problem as well as the real noisy image denoising problem.

我们的去噪模型是:
\begin{equation}
\min_{\mathbf{D},\mathbf{C},\mathbf{W}}\frac{1}{2}\|(\mathbf{Y}-\mathbf{D}\mathbf{C})\mathbf{W}\|_{F}^{2}
+
\lambda\|\mathbf{C}\|_{1}
\quad
\text{s.t.}
\quad
\mathbf{D}^{\top}\mathbf{D} =\mathbf{I}. 
\end{equation}

去噪过程如下:

1. 初始化:

我们从原始噪声图得到相似块矩阵$\mathbf{Y}$,我们采用\cite{Chen2015ICCV}的方法估计彩色带噪图的噪声水平$\sigma_{0}$。初始化权重矩阵$\mathbf{W}^{(0)}=\frac{1}{\sigma_{0}}\mathbf{I}$, 初始化字典$\mathbf{D}^{(0)}=\mathbf{I}$。

2. 进入内部迭代优化:

对于每一次迭代,模型都需要反复迭代求解$\mathbf{D}$,$\mathbf{C}$直到收敛。For $k=0,1,2,...$:

a. update $\mathbf{C}$
\begin{equation}
\min_{\mathbf{C}}\frac{1}{2}\|(\mathbf{Y}-\mathbf{D}^{(k)}\mathbf{C})\mathbf{W}^{(k)}\|_{F}^{2}
+
\lambda\|\mathbf{C}\|_{1}.
\end{equation}
有闭合解,每一列单独求解:
\begin{equation}
(\hat{\mathbf{c}}_{i})^{(k+1)}
=
\arg\min_{\mathbf{c}_{i}}\frac{1}{2}\|(\mathbf{y}_{i}-\mathbf{D}^{(k)}\mathbf{c}_{i})\mathbf{W}_{ii}\|_{2}^{2}
+
\lambda\|\mathbf{c}_{i}\|_{1}.
\end{equation}
闭合解为:
\begin{equation}
(\hat{\mathbf{c}}_{i})^{(k+1)}
=
\text{sgn}(\mathbf{D^{\top}y}) 
\odot 
\text{max}(|\mathbf{D^{\top}y}|-\frac{\lambda}{(\mathbf{W}_{ii})^{2}},0),
\end{equation}

b. update $\mathbf{D}$
\begin{equation}
\min_{\mathbf{D}}\frac{1}{2}\|(\mathbf{Y}-\mathbf{D}\mathbf{C}^{(k+1)})\mathbf{W}\|_{F}^{2}
\quad
\text{s.t.}
\quad
\mathbf{D}^{\top}\mathbf{D} =\mathbf{I}. 
\end{equation}
等价于
\begin{equation}
\min_{\mathbf{D}}\|(\mathbf{Y}\mathbf{W})-\mathbf{D}(\mathbf{C}^{(k+1)}\mathbf{W})\|_{F}^{2}
\quad
\text{s.t.}
\quad
\mathbf{D}^{\top}\mathbf{D} = \mathbf{I},
\end{equation}
闭合解为:$\hat{\mathbf{D}}^{(k+1)}=\mathbf{V}\mathbf{U}^{\top}$, $\mathbf{C}\mathbf{W}(\mathbf{Y}\mathbf{W})^{\top}=\mathbf{U}\mathbf{\Sigma}\mathbf{V}^{\top}$.

c. update $\mathbf{W}$
根据贝叶斯法则,权重矩阵的第$i$项为
\begin{equation}
\mathbf{W}_{ii} 
=\frac{\frac{1}{N}\sum_{i=1}^{N}\|\mathbf{y}_{i}-\mathbf{D}\mathbf{c}_{i}\|_{2}}{\sigma_{\mathbf{y}_{i}}\|\mathbf{y}_{i}-\mathbf{D}\mathbf{c}_{i}\|_{2}}
\end{equation}


3. 外部迭代优化:

更新每个块的噪声水平:
\begin{equation}
\sigma_{\mathbf{y}_{i}} = \sqrt{\sigma_{0}^{2} - \|\mathbf{y}_{i}-\mathbf{D}\mathbf{c}_{i}\|_{2}^{2}}
\end{equation}
然后重复步骤2.


{
\small
\bibliographystyle{unsrt}
\bibliography{egbib}
}

\end{document}