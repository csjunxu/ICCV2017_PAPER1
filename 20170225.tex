
\documentclass[10pt,twocolumn,letterpaper]{article}

\usepackage{iccv}
\usepackage{times}
\usepackage{epsfig}
\usepackage{graphicx}
\usepackage{amsmath}
\usepackage{amssymb}
\usepackage[numbers,sort]{natbib}
\usepackage[UTF8]{ctex}



% Include other packages here, before hyperref.

% If you comment hyperref and then uncomment it, you should delete
% egpaper.aux before re-running latex.  (Or just hit 'q' on the first latex
% run, let it finish, and you should be clear).
\usepackage[pagebackref=true,breaklinks=true,letterpaper=true,colorlinks,bookmarks=false]{hyperref}

% \iccvfinalcopy % *** Uncomment this line for the final submission

\def\iccvPaperID{****} % *** Enter the ICCV Paper ID here
\def\httilde{\mbox{\tt\raisebox{-.5ex}{\symbol{126}}}}

% Pages are numbered in submission mode, and unnumbered in camera-ready
\ificcvfinal\pagestyle{empty}\fi
\begin{document}

%%%%%%%%% TITLE
\title{\LaTeX\ Author Guidelines for ICCV Proceedings}

\author{First Author\\
Institution1\\
Institution1 address\\
{\tt\small firstauthor@i1.org}
% For a paper whose authors are all at the same institution,
% omit the following lines up until the closing ``}''.
% Additional authors and addresses can be added with ``\and'',
% just like the second author.
% To save space, use either the email address or home page, not both
\and
Second Author\\
Institution2\\
First line of institution2 address\\
{\tt\small secondauthor@i2.org}
}

\maketitle
%\thispagestyle{empty}


\section{Introduction}

我们的去噪模型是:
\begin{equation}
\min_{\mathbf{D},\mathbf{C},\mathbf{W}}\frac{1}{2}\|(\mathbf{Y}-\mathbf{D}\mathbf{C})\mathbf{W}\|_{F}^{2}
+
\lambda\|\mathbf{C}\|_{1}
\quad
\text{s.t.}
\quad
\mathbf{D}^{\top}\mathbf{D} =\mathbf{I}. 
\end{equation}

去噪过程如下:

1. 初始化:

我们从原始噪声图得到相似块矩阵$\mathbf{Y}$,我们采用\cite{Chen2015ICCV}的方法估计彩色带噪图的噪声水平$\sigma_{0}$。初始化权重矩阵$\mathbf{W}^{(0)}=\frac{1}{\sigma_{0}}\mathbf{I}$, 初始化字典$\mathbf{D}^{(0)}=\mathbf{I}$。

2. 进入内部迭代优化:

对于每一次迭代,模型都需要反复迭代求解$\mathbf{D}$,$\mathbf{C}$直到收敛。For $k=0,1,2,...$:

a. update $\mathbf{C}$
\begin{equation}
\min_{\mathbf{C}}\frac{1}{2}\|(\mathbf{Y}-\mathbf{D}^{(k)}\mathbf{C})\mathbf{W}^{(k)}\|_{F}^{2}
+
\lambda\|\mathbf{C}\|_{1}.
\end{equation}
有闭合解,每一列单独求解:
\begin{equation}
(\hat{\mathbf{c}}_{i})^{(k+1)}
=
\arg\min_{\mathbf{c}_{i}}\frac{1}{2}\|(\mathbf{y}_{i}-\mathbf{D}^{(k)}\mathbf{c}_{i})\mathbf{W}_{ii}\|_{2}^{2}
+
\lambda\|\mathbf{c}_{i}\|_{1}.
\end{equation}
闭合解为:
\begin{equation}
(\hat{\mathbf{c}}_{i})^{(k+1)}
=
\text{sgn}(\mathbf{D^{\top}y}) 
\odot 
\text{max}(|\mathbf{D^{\top}y}|-\frac{\lambda}{(\mathbf{W}_{ii})^{2}},0),
\end{equation}

b. update $\mathbf{D}$
\begin{equation}
\min_{\mathbf{D}}\frac{1}{2}\|(\mathbf{Y}-\mathbf{D}\mathbf{C}^{(k+1)})\mathbf{W}\|_{F}^{2}
\quad
\text{s.t.}
\quad
\mathbf{D}^{\top}\mathbf{D} =\mathbf{I}. 
\end{equation}
等价于
\begin{equation}
\min_{\mathbf{D}}\|(\mathbf{Y}\mathbf{W})-\mathbf{D}(\mathbf{C}^{(k+1)}\mathbf{W})\|_{F}^{2}
\quad
\text{s.t.}
\quad
\mathbf{D}^{\top}\mathbf{D} = \mathbf{I},
\end{equation}
闭合解为:$\hat{\mathbf{D}}^{(k+1)}=\mathbf{V}\mathbf{U}^{\top}$, $\mathbf{C}\mathbf{W}(\mathbf{Y}\mathbf{W})^{\top}=\mathbf{U}\mathbf{\Sigma}\mathbf{V}^{\top}$.

c. update $\mathbf{W}$
根据贝叶斯法则,权重矩阵的第$i$项为
\begin{equation}
\mathbf{W}_{ii} 
=\frac{\sqrt{d}\frac{1}{N}\sum_{i=1}^{N}\|\mathbf{y}_{i}-\mathbf{D}\mathbf{c}_{i}\|_{2}}{\sigma_{\mathbf{y}_{i}}\|\mathbf{y}_{i}-\mathbf{D}\mathbf{c}_{i}\|_{2}}
\end{equation}


3. 外部迭代优化:

更新每个块的噪声水平:
\begin{equation}
\sigma_{\mathbf{y}_{i}} = \sqrt{\sigma_{0}^{2} - \|\mathbf{y}_{i}-\mathbf{D}\mathbf{c}_{i}\|_{2}^{2}}
\end{equation}
然后重复步骤2.


{
\small
\bibliographystyle{unsrt}
\bibliography{egbib}
}

\end{document}