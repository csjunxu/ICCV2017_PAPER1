
\documentclass[10pt,twocolumn,letterpaper]{article}

\usepackage{iccv}
\usepackage{times}
\usepackage{epsfig}
\usepackage{graphicx}
\usepackage{amsmath}
\usepackage{amssymb}
\usepackage[numbers,sort]{natbib}
\usepackage[UTF8]{ctex}



% Include other packages here, before hyperref.

% If you comment hyperref and then uncomment it, you should delete
% egpaper.aux before re-running latex.  (Or just hit 'q' on the first latex
% run, let it finish, and you should be clear).
\usepackage[pagebackref=true,breaklinks=true,letterpaper=true,colorlinks,bookmarks=false]{hyperref}

% \iccvfinalcopy % *** Uncomment this line for the final submission

\def\iccvPaperID{****} % *** Enter the ICCV Paper ID here
\def\httilde{\mbox{\tt\raisebox{-.5ex}{\symbol{126}}}}

% Pages are numbered in submission mode, and unnumbered in camera-ready
\ificcvfinal\pagestyle{empty}\fi
\begin{document}

%%%%%%%%% TITLE
\title{\LaTeX\ Author Guidelines for ICCV Proceedings}

\author{First Author\\
Institution1\\
Institution1 address\\
{\tt\small firstauthor@i1.org}
% For a paper whose authors are all at the same institution,
% omit the following lines up until the closing ``}''.
% Additional authors and addresses can be added with ``\and'',
% just like the second author.
% To save space, use either the email address or home page, not both
\and
Second Author\\
Institution2\\
First line of institution2 address\\
{\tt\small secondauthor@i2.org}
}

\maketitle
%\thispagestyle{empty}


\section{Introduction}

假设noisy image的生成模型是:
\begin{equation}
y(\mu)=x(\mu) + n(\mu)
\end{equation}
其中$\mu$是图像中的像素位置,并且假定$n(\mu)\sim \mathcal{N}(0,\sigma^{2}({\mu}))$,假设$\sigma({\mu})\in C^{1}(\Omega)$,即噪声的标准差是一个光滑函数。
模拟实验中,我们可以假设$\sigma({\mu})\in [\frac{1}{1+\epsilon}\sigma, (1+\epsilon)\sigma]$。

我们的去噪模型是:
\begin{equation}
\min_{\mathbf{D},\mathbf{C}}\frac{1}{2}\|(\mathbf{Y}-\mathbf{D}\mathbf{C})\mathbf{W}\|_{F}^{2}
+
\lambda\sigma_{Y}^{2}\|\mathbf{C}\|_{1}
\quad
\text{s.t.}
\quad
\mathbf{D}^{\top}\mathbf{D} =\mathbf{I}. 
\end{equation}

去噪过程如下:
\\
1. 第一次迭代中,我们从原始噪声图得到相似块矩阵$\mathbf{Y}$,我们采用\cite{Chen2015ICCV}的方法估计彩色带噪图的噪声水平$\sigma_{0}$,$\mathbf{W}$也是全1矩阵。
用上述模型去噪之后,得到第一次去噪后的相似块矩阵$\mathbf{X}=\mathbf{DC}$。
\\
2. 从第二次迭代开始,给定上一次迭代得到的非局部相似块矩阵$\mathbf{Y}$,我们从中估计$\mathbf{W}$(以下只是某种形式之一,需根据实验结果调整):
\begin{equation}
\mathbf{W}_{ii} 
=
\exp(-\tau_{1}\|\mathbf{y}_{ii}-\mathbf{D}\mathbf{c}_{ii}\|_{2}^{2})
*\exp(-\tau_{2}\|\mathbf{y}_{1}-\mathbf{y}_{ii}\|_{2}^{2}).
\end{equation}
$\mathbf{W}_{ii}$的设计原理是:$\mathbf{W}_{ii}$与上次迭代里,算法对第$i$个相似块$\mathbf{y}_{ii}$去掉的噪声的多少有关($\exp(-\tau_{1}\|\mathbf{y}_{ii}-\mathbf{D}\mathbf{c}_{ii}\|_{2}^{2})$),并且与第$i$个相似块和第一个种子块的欧式距离有关($\exp(-\tau_{2}\|\mathbf{y}_{1}-\mathbf{y}_{ii}\|_{2}^{2})$)。我们希望设计出一个框架,使得$\mathbf{W}_{ii}$可以自动探测第$i$个块里的噪声还有多少,从而可以有效去噪真实的噪声图。

当然,不同的相似块有不同的噪声水平。从而对于$\mathbf{Y}$的正则项参数之一$\sigma_{Y}$需要根据$\mathbf{Y}$估计得到,比如有:
\begin{equation}
\sigma_{Y} = f(\mathbf{Y},\mathbf{D},\mathbf{C})
\end{equation}
一个简单的例子是:
\begin{equation}
\sigma_{Y} = \sqrt{\sigma_{0}^{2}- \frac{1}{N}(\sum_{i=1}^{N}\|\mathbf{y}_{i}-\mathbf{D}\mathbf{c}_{i}\|_{2}^{2})}
\end{equation}
即初始估计的噪声方差$-$相似块矩阵的N个块里去掉的噪声方差,再开根号。

3. 对于每一次迭代,模型都需要反复迭代求解$\mathbf{D}$,$\mathbf{C}$直到收敛。For $k=0,1,2,...$:

a. update $\mathbf{C}$
\begin{equation}
\min_{\mathbf{C}}\frac{1}{2}\|(\mathbf{Y}-\mathbf{D}^{(k)}\mathbf{C})\mathbf{W}\|_{F}^{2}
+
\lambda\|\mathbf{C}\|_{1}.
\end{equation}
有闭合解,每一列单独求解:
\begin{equation}
(\hat{\mathbf{c}}_{i})^{(k+1)}
=
\arg\min_{\mathbf{c}_{i}}\frac{1}{2}\|(\mathbf{y}_{i}-\mathbf{D}^{(k)}\mathbf{c}_{i})\mathbf{W}_{ii}\|_{2}^{2}
+
\lambda\|\mathbf{c}_{i}\|_{1}.
\end{equation}
闭合解为:
\begin{equation}
(\hat{\mathbf{c}}_{i})^{(k+1)}
=
\text{sgn}(\mathbf{D^{\top}y}) 
\odot 
\text{max}(|\mathbf{D^{\top}y}|-\frac{\lambda}{(\mathbf{W}_{ii})^{2}},0),
\end{equation}

b. update $\mathbf{D}$
\begin{equation}
\min_{\mathbf{D}}\frac{1}{2}\|(\mathbf{Y}-\mathbf{D}\mathbf{C}^{(k+1)})\mathbf{W}\|_{F}^{2}
\quad
\text{s.t.}
\quad
\mathbf{D}^{\top}\mathbf{D} =\mathbf{I}. 
\end{equation}
等价于
\begin{equation}
\min_{\mathbf{D}}\|(\mathbf{Y}\mathbf{W})-\mathbf{D}(\mathbf{C}^{(k+1)}\mathbf{W})\|_{F}^{2}
\quad
\text{s.t.}
\quad
\mathbf{D}^{\top}\mathbf{D} = \mathbf{I},
\end{equation}
闭合解为:$\hat{\mathbf{D}}^{(k+1)}=\mathbf{V}\mathbf{U}^{\top}$, $\mathbf{C}\mathbf{W}(\mathbf{Y}\mathbf{W})^{\top}=\mathbf{U}\mathbf{\Sigma}\mathbf{V}^{\top}$.

{
\small
\bibliographystyle{unsrt}
\bibliography{egbib}
}

\end{document}