
\documentclass[10pt,twocolumn,letterpaper]{article}

\usepackage{iccv}
\usepackage{times}
\usepackage{epsfig}
\usepackage{graphicx}
\usepackage{amsmath}
\usepackage{amssymb}

\usepackage[UTF8]{ctex}



% Include other packages here, before hyperref.

% If you comment hyperref and then uncomment it, you should delete
% egpaper.aux before re-running latex.  (Or just hit 'q' on the first latex
% run, let it finish, and you should be clear).
\usepackage[pagebackref=true,breaklinks=true,letterpaper=true,colorlinks,bookmarks=false]{hyperref}

% \iccvfinalcopy % *** Uncomment this line for the final submission

\def\iccvPaperID{****} % *** Enter the ICCV Paper ID here
\def\httilde{\mbox{\tt\raisebox{-.5ex}{\symbol{126}}}}

% Pages are numbered in submission mode, and unnumbered in camera-ready
\ificcvfinal\pagestyle{empty}\fi
\begin{document}

%%%%%%%%% TITLE
\title{\LaTeX\ Author Guidelines for ICCV Proceedings}

\author{First Author\\
Institution1\\
Institution1 address\\
{\tt\small firstauthor@i1.org}
% For a paper whose authors are all at the same institution,
% omit the following lines up until the closing ``}''.
% Additional authors and addresses can be added with ``\and'',
% just like the second author.
% To save space, use either the email address or home page, not both
\and
Second Author\\
Institution2\\
First line of institution2 address\\
{\tt\small secondauthor@i2.org}
}

\maketitle
%\thispagestyle{empty}


%%%%%%%%% ABSTRACT
\begin{abstract}
   The ABSTRACT is to be in fully-justified italicized text, at the top
   of the left-hand column, below the author and affiliation
   information. Use the word ``Abstract'' as the title, in 12-point
   Times, boldface type, centered relative to the column, initially
   capitalized. The abstract is to be in 10-point, single-spaced type.
   Leave two blank lines after the Abstract, then begin the main text.
   Look at previous ICCV abstracts to get a feel for style and length.
	 Please note that the title can be up to 512 characters in length. 
	 The maximum size of the abstract is 4000 characters.
\end{abstract}

%%%%%%%%% BODY TEXT
\section{Introduction}
我们的模型主体是这样的:
\begin{equation}
\min_{\mathbf{D},\mathbf{C}}\frac{1}{2}\|(\mathbf{Y}-\mathbf{D}\mathbf{C})\mathbf{W}\|_{F}^{2}
+
\lambda\|\mathbf{C}\|_{1,s}
\quad
\text{s.t.}
\quad
\mathbf{D}^{\top}\mathbf{D} =\mathbf{I}. 
\end{equation}
其中$\|\mathbf{C}\|_{1,s}$表示用的是加权L1范数。
\\
\\
a. 初始化:
\\
$\mathbf{D}^{(0)}$由$\mathbf{Y}=\mathbf{D}\mathbf{\Sigma}^{(0)}\mathbf{V}^{\top}$做SVD分解得到;
\\
$\mathbf{s}^{(0)}=\text{diag}(\mathbf{\Sigma}^{(0)})$是weighted sparse coding的权重,每一行的值不同;
\\
$\mathbf{W}^{(0)}$是对角矩阵,初始化为恒等矩阵,原因是一开始就平等对待;或者对每个$\mathbf{Y}$的每一列(图像块)估计一个噪声水平$\sigma$,作为$\mathbf{W}^{(0)}$对角上的相应元素。
\\
\\
\\
b. 进入迭代求解$\mathbf{D}$,$\mathbf{C}$,并更新$\mathbf{W}$ ($k=0,1,2,...$):
\\
\subsection{update $\mathbf{C}$}
\begin{equation}
\mathbf{C}^{(k+1)}
=
\arg\min_{\mathbf{C}}\frac{1}{2}\|(\mathbf{Y}-\mathbf{D}^{(k)}\mathbf{C})\mathbf{W}^{(k)}\|_{F}^{2}
+
\lambda\|\mathbf{C}\|_{1,s}.
\end{equation}
有闭合解,每一列单独求解:
\begin{equation}
\mathbf{c}_{i}^{(k+1)}
=
\arg\min_{\mathbf{c}_{i}}\frac{1}{2}\|(\mathbf{y}_{i}-\mathbf{D}^{(k)}\mathbf{c}_{i})\mathbf{W}_{ii}^{(k)}\|_{2}^{2}
+
\lambda\|\mathbf{s}^{T}\mathbf{c}_{i}\|_{1}.
\end{equation}
闭合解为:
\begin{equation}
\mathbf{c}_{i}^{(k+1)} 
=
\text{sgn}(\mathbf{D^{\top}y}) 
\odot 
\text{max}(|\mathbf{D^{\top}y}|-\frac{\lambda}{(\mathbf{W}_{ii}^{(k)})^{2}}\mathbf{s},0),
\end{equation}



\subsection{update $\mathbf{D}$}
\begin{equation}
\begin{split}
\mathbf{D}^{(k+1)}
&
=
\arg\min_{\mathbf{D}}\frac{1}{2}\|(\mathbf{Y}-\mathbf{D}\mathbf{C}^{(k+1)})\mathbf{W}^{(k)}\|_{F}^{2}
\\
&
\quad
\text{s.t.}
\quad
\mathbf{D}^{\top}\mathbf{D} =\mathbf{I}. 
\end{split}
\end{equation}
闭合解为:
\begin{equation}
\mathbf{D}^{(k+1)}
=
\mathbf{U}\mathbf{V}^{\top}
\end{equation}
其中,
$\mathbf{U}$和$\mathbf{V}$由$\mathbf{C}^{(k+1)}(\mathbf{W}^{(k)})^{2}\mathbf{Y}^{\top}=\mathbf{U}\mathbf{\Sigma}^{(k+1)}\mathbf{V}^{\top}
$
做SVD分解得到。

\subsection{update $\mathbf{s}$}
$\mathbf{s}^{k+1}=\mathbf{\Sigma}^{(k+1)}$

\subsection{update $\mathbf{W}$}
$\mathbf{W}$是对角矩阵,只需要更新迭代对角元即可,:
\begin{equation}
\mathbf{W}_{ii}^{k+1}=\exp(-\lambda_{2}\|\mathbf{y}_{i}-\mathbf{D}^{(k+1)}\mathbf{c}_{i}^{(k+1)}\|_{2})
\end{equation}
$\lambda_{2}$是参数。

待解决问题:
需要探讨收敛性,收敛速率等问题,收敛值的上下界等问题。
实际中,我没有多次迭代,因为多次迭代速度太慢,权重的设计与效果也值得深入探讨。


\section{孟老师的模型以及区别}
孟老师的模型:
\begin{equation}
\min_{\mathbf{U},\mathbf{V}}\|\mathbf{W}\odot(\mathbf{X}-\mathbf{U}\mathbf{V}^{\top})\|_{1}
\end{equation}
孟老师的模型是coordinate-wise的迭代,其不断迭代$\mathbf{U},\mathbf{V}$直到收敛。
这个模型应用再Low-Rank Matrix Factorization with Missing Entries这个问题里。

区别:
\begin{itemize}
\item 模型不同;
\item 应用不同;
\item 权重的设计和迭代方式不同,物理含义不一样。孟老师的模型里的权重主要是指矩阵的元素是否是missing entry,也就是$\mathbf{W}$里只有0或1.
\end{itemize}

\section{杨猛的模型以及区别}
杨猛的模型:
\begin{equation}
\min_{\mathbf{c}}\|\mathbf{W}^{\frac{1}{2}}(\mathbf{y}-\mathbf{D}\mathbf{c})\|_{2}^{2}
\quad
\text{s.t.}
\quad
\|\mathbf{c}\|_{1}<\sigma
\end{equation}

区别:
\begin{itemize}
\item 具体到加权,我的方法里,F范数的权重是加在列(sample)上,但是杨猛是加在F范数的每一行(variable)上;而且设计出发点不同(我的模型是从weighted orthogonal Procrustes problem里的模型里演化过来的),权重迭代方式不同;
\item 我的模型里,字典是正交的,是通过数据学到的,有闭合解,杨的方法是训练集里的sample直接做字典;
\item 我的模型里系数1范的权重是加在每一行上,所以,相当于我的模型里,每行每列的每个元素都带权重;而且系数有闭合解。
\end{itemize}



{\small
\bibliographystyle{ieee}
\bibliography{egbib}
}

\end{document}
